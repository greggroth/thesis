\tikzstyle{data} = [draw=none, fill=goldfish]
\tikzstyle{temptools} = [draw=none, fill=aoi]
\tikzstyle{spm} = [draw=none, fill=beachstorm]
\tikzstyle{params} = [draw=none, fill=pondwater]
\tikzstyle{line} = [draw, very thick, color=black!50, -latex']

\begin{tikzpicture}[node distance=0.7cm, rectangle, text width=4.5cm, text badly centered, rounded corners, minimum height=1cm, anchor=north]     
  % Left column
    \node[data](fmridata){fMRI BOLD Data};
    \node[temptools, below=of fmridata](calcrest){Calculate resting state (avg\_NII\_rest)};
    \node[temptools, below=of calcrest](normalize){Normalize the data to resting state (avg\_NII\_normalize)};
    \node[temptools, below=of normalize](boldtomf){Calculate the change in metabolism and blood flow (BOLDtoMF)\\Details given in \cref{sec:calcmf}};
    % middle column
    \node[data, right=of fmridata](t1contrast){T1 contrast image};
    \node[spm, below=of t1contrast](segment){Segment image (SPM8)};
    \node[temptools, right=of boldtomf](buildhead){Build head matrix (ImportSegmentedT1)\\Details given in \cref{sec:prephead}};
    % right column
    \node[temptools, right=of buildhead](calcequil){Calculate equilibrium temperature (tempCalcEquilibrium)\\Details given in \cref{sec:calcequilT}};
    \node[params, above=of calcequil, xshift=-2cm](tissueparams){Tissue-specific parameters (given in \cref{tbl:tissues})};
    % bottom
    \node[temptools, below=of buildhead, text width=8cm](calctemp){Find temperature change during activity (tempCalcDynMF)\\Details given in \cref{sec:calcT}};
  
  \path[line](fmridata) -- (calcrest);
  \path[line](calcrest) -- (normalize);
  \path[line](normalize) -- (boldtomf);
  \path[line](t1contrast) -- (segment);
  \path[line](segment) -- (buildhead);
  \path[line](buildhead) -- (calcequil);
  \path[line](boldtomf) |- (calctemp);
  \path[line](buildhead) -- (calctemp);
  \path[line](calcequil) |- (calctemp); 
  \path[line](tissueparams) -| (buildhead);
\end{tikzpicture}