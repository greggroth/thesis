\thispagestyle{empty}
\begin{center}
  { \huge \thesisTitle } \\
  \vspace{0.5in}
  by \\
  \vspace{0.5in}
  { \large \textsc{Greggory H. Rothmeier} } \\
  \vspace{0.5in}
  Under the Direction of A. G. Unil Perera\\
  \vspace{0.5in}
  \textsc{Abstract}
  \vspace{0.2in}
\end{center}
  \begin{quote}
    Regional tissue temperature dynamics in the brain is determined by the balance of the metabolic heat production rate 
    and heat exchange with blood flowing through capillaries embedded in the tissue, the surrounding tissues and the environment. Local 
    changes in blood flow and metabolism during functional activity can upset this balance and induce transient temperature changes. 
    Invasive experimental studies in animal models have established that the brain temperature changes during functional activity are observable 
    and a definitive relationship exists between temperature and brain activity. Here, we present a theoretical framework that links tissue 
    temperature dynamics with hemodynamic activity allowing us to nonivasively estimate brain temperature changes from experimentally 
    measured blood-oxygen level dependent (BOLD) signals. With this unified approach, we are able to pinpoint the mechanisms for 
    hemodynamic activity-related temperature increases and decreases. Applying this approach to experimentally measured fMRI timeseries 
    recorded in a finger-movement task, we also provide an estimate of the temperature change during a functional activity in the brain.
  \end{quote}
\vspace*{\fill}
INDEX WORDS: fMRI, BOLD, Temperature, etc.