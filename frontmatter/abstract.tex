\thispagestyle{empty}
\begin{center}
  \normalsize
  \textsc{\thesisTitle } \\
  \vspace{0.2in}
  by \\
  \vspace{0.2in}
  \textsc{Greggory H. Rothmeier} \\
  \vspace{0.2in}
  Under the Direction of A. G. Unil Perera\\
  \vspace{0.2in}
  \textsc{Abstract}
  \vspace{0.2in}
\end{center}
    Regional tissue temperature dynamics in the brain is determined by the balance of the metabolic heat production rate and heat exchange with blood flowing through capillaries embedded in the tissue, the surrounding tissues and the environment. Local changes in blood flow and metabolism during functional activity can upset this balance and induce transient temperature changes. Invasive experimental studies in animal models have established that the brain temperature changes during functional activity are observable and a definitive relationship exists between temperature and brain activity. We present a theoretical framework that links tissue temperature dynamics with hemodynamic activity allowing us to non-invasively estimate brain temperature changes from experimentally measured blood-oxygen level dependent (BOLD) signals. With this unified approach, we are able to pinpoint the mechanisms for hemodynamic activity-related temperature increases and decreases.  In addition to this, the potential uses and limitations of optical measurements are discussed.  
\vspace*{\fill}\\
INDEX WORDS: Functional magnetic resonance imaging, Blood oxygen level dependent, Temperature, Functional near-infrared spectroscopy