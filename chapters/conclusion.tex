\chapter{Conclusion}

It has been shown that by considering the entire head within the model, brain temperature can be reliably calculated from non-invasive fMRI measurements. Experimental measurements of activity-induced brain temperature changes have shown that a simple relationship does not exist~\citep{mcelligott,kiyatkin,zeschke,george,tachibana}. Single-voxel brain temperature modeling efforts predict that an increase in brain activity will induce a decrease in temperature. This one-dimensional perspective does not account for the spatial distribution of heat throughout the head like a multi-voxel approach.

Our model of brain temperature changes is able to account for the variability found in experimental brain temperature measurements.  This is accomplished by modeling heat dynamics throughout the entire head rather than reducing the model to one ROI.  It was found that the variability in experiment measurements is most likely due to differences in resting state temperatures throughout the brain.  Since each voxel is at a slightly different temperature, the same change in the BOLD response may result in different changes in temperature. Additionally, it was found through the model that a thin (4--6 mm) region of outer cortical tissue is at a resting temperature below the blood temperature.  In this region, an increase in brain activity (inducing an increase in CBF) will warm the tissue.  Thus, with the same BOLD response, tissue may either be warmed or cooled depending on it's proximity to the surface of the head.

The biggest shortcoming of our model is that we are unable to independently compare our calculations with experimental measurements of temperature and BOLD response. It was not possible for us to do this because there is currently not a method for non-invasively measuring temperature independent of an fMRI. An improvement to the model could also be gained by a more accurate method of CMRO$_2$ and CBF calculations from the BOLD response. The current method uses empirically fit formulas, so the accuracy is limited by the data used for the fitting. A model that does not rely on experimental data would be ideal. The calculations could also be improved by segmenting the head into more tissue types.  We used six tissue types, but the use of more would further improve the calculations since each tissue type has different physiological parameters (thermal conduction, baseline heat production, etc.). A separate line of research that could be pursued would be a model for calculating brain temperature changes from fNIRS data. Both fNIRS and fMRI BOLD response detect changes in local tissue oxygenation, so it should be possible to adapt our model to use fNIRS data. If such a model existed, calculations from it and our model could be compared to refine both models.

Although it is expected that the contribution would be negligible~\citep{nadel1971}, our model does not take into account the effects of perspiration. It would likely not affect the change in temperature greatly because it takes place a couple of centimeters away from brain tissue.  Another physiological affect not account for is temperature regulation by the pre-optic nucleus of the anterior hypothalamus~\citep{bertolizio2011}. It is responsible for balancing heat production and dissipation~\citep{simon1993} and if the model were applied to cases where extreme brain temperatures are created then it would be important to account for how this would react.

How human brain temperature is affected by changes in local brain activity is not well understood because the changes are small and current experimental measurement techniques may require invasive procedures. Models such as the one proposed here allow for brain temperature to be understood through non-invasive measurements such as the fMRI BOLD response.

